The purpose of the dashboard is to support exploratory discovery of edible plant observations in Denmark.
Users typically do not have predefined questions which makes the exploratory analysis important.

The visualization is needed to give insights into patterns, trends, and distributions that cannot be identified using a table alone.

The dashboard supports several actions and targets. 

\item Users \textbf{discover} spatial and temporal \textbf{trends} to gain an overview of where observations are concentrated.
\item Users \textbf{explore} spatial \textbf{features} to idenity regions with high or low observation density.
\item Users \textbf{compare} \textbf{distributions} of observations across regions, time periods, and plant categories.
\item Users \textbf{summarize} temporal \textbf{distributions} of observations to understand how observation activity varies over time.
\item Users \textbf{locate} known species when their \textbf{spatial location} is unknown.
\item Users \textbf{browse} \textbf{attributes of observations} within a selected region when species are unknown.