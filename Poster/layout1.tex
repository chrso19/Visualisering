%%
%% This is file `tikzposter-example.tex',
%% generated with the docstrip utility.
%%
%% The original source files were:
%%
%% tikzposter.dtx  (with options: `tikzposter-example.tex')
%% 
%% This is a generated file.
%% 
%% Copyright (C) 2014 by Pascal Richter, Elena Botoeva, Richard Barnard, and Dirk Surmann
%% 
%% This file may be distributed and/or modified under the
%% conditions of the LaTeX Project Public License, either
%% version 2.0 of this license or (at your option) any later
%% version. The latest version of this license is in:
%% 
%% http://www.latex-project.org/lppl.txt
%% 
%% and version 2.0 or later is part of all distributions of
%% LaTeX version 2013/12/01 or later.
%% 


\documentclass[20pt, a1paper, portrait, colspace = 0.5cm, innermargin = 0.6cm,
blockverticalspace = 0.7cm]{tikzposter}



% Core colors
\definecolor{mainbg}{HTML}{F7F3E9}           % cream
\definecolor{textcolor}{HTML}{0B0B09}        % onyx
\definecolor{blockcolor}{HTML}{628B48}       % fern
\definecolor{titleblockcolor}{HTML}{053913}  % black-forest

% Accent colors
\definecolor{accentone}{HTML}{BAB823} % flower
\definecolor{accenttwo}{HTML}{690C4A}   % Berry
\definecolor{accentthree}{HTML}{9E1210}   % Fruit




\definecolorstyle{myColorStyle}{
\colorlet{colorOne}{mainbg}
\colorlet{colorTwo}{blockcolor}
\colorlet{colorThree}{titleblockcolor}
}{
% Background Color
\colorlet{backgroundcolor}{colorTwo!20}
% Title Colors
\colorlet{titlefgcolor}{colorOne}
\colorlet{titlebgcolor}{colorThree}
\colorlet{framecolor}{colorThree!70}
% Block Colors
\colorlet{blocktitlebgcolor}{colorThree!80}
\colorlet{blocktitlefgcolor}{colorOne}
\colorlet{blockbodybgcolor}{colorTwo!30}
\colorlet{blockbodyfgcolor}{black}
% Innerblock Colors
\colorlet{innerblocktitlebgcolor}{white}
\colorlet{innerblocktitlefgcolor}{colorOne}
\colorlet{innerblockbodybgcolor}{white}
\colorlet{innerblockbodyfgcolor}{colorOne}
% Note colors
\colorlet{notefgcolor}{black}
\colorlet{notebgcolor}{white}
\colorlet{notefrcolor}{white}
}


\usecolorstyle{myColorStyle}
\usetitlestyle{Default}
\useblockstyle[titleinnersep = 0.5cm]{Default}
\tikzposterlatexaffectionproofoff
\usetikzlibrary{positioning}
\usepackage{amssymb,amsmath}
\usepackage[table]{xcolor}
\usepackage{tabularx}

\usepackage{blindtext}
\usepackage{caption}
\usepackage{subcaption}
\newcounter{subfigcount}
\newcommand{\mysubcaption}[1]{%
    \vskip 0.5cm {\small \textbf{(\alph{subfigcount})} #1}%
    \stepcounter{subfigcount}%
}

\setlength{\extrarowheight}{3pt}


\usepackage{environ}
\setlength{\tabcolsep}{10pt} % default er 6pt

%% --- Table formatting/style ---

\newsavebox{\tablebox}

\NewEnviron{rndtable}[1]{% #1 = tabularkolonner
  \addtolength{\extrarowheight}{8pt}%
  \savebox{\tablebox}{%
    \begin{tabular}{#1}%
      \BODY%
    \end{tabular}}%
  \begin{tikzpicture}
    \begin{scope}
      % Baggrundsfarve bag hele tabellen
      \fill[colorTwo!90, rounded corners=1ex]
        (0,-\dp\tablebox) rectangle (\wd\tablebox,\ht\tablebox);
      % Klip tabellen til afrundede hjørner
      \clip[rounded corners=1ex]
        (0,-\dp\tablebox) rectangle (\wd\tablebox,\ht\tablebox);
      % Indsæt tabelindholdet
      \node at (0,-\dp\tablebox)
        [anchor=south west,inner sep=0pt]{\usebox{\tablebox}};
    \end{scope}
  \end{tikzpicture}
}

%% --- Figure formatting/style---
\newsavebox{\figbox}

\NewEnviron{rndfig}[1][]% Optional TikZ options
{%
  \savebox{\figbox}{\BODY}%
  \begin{tikzpicture}[#1]
    \begin{scope}
      % Baggrundsform med afrundede hjørner
      \fill[colorTwo!90, rounded corners=1ex]
        (0,-\dp\figbox) rectangle (\wd\figbox,\ht\figbox);
      % Klip figuren til afrundede hjørner
      \clip[rounded corners=1ex]
        (0,-\dp\figbox) rectangle (\wd\figbox,\ht\figbox);
      % Indsæt figur
      \node at (0,-\dp\figbox)
        [anchor=south west,inner sep=0pt]{\usebox{\figbox}};
    \end{scope}
  \end{tikzpicture}
}


 % Title, Author, Institute
 \title{\parbox{\linewidth}{\centering Visual Exploration of Edible Wild Flora:\\ \LARGE A foraging-Oriented Map Tool\\[1cm] }}
 \author{\large Kessor Chao, Christine Sandager Søgaard \& Anna Ellen Vedel Braaby}
 \institute{\large University of Southern Denmark, Odense, Denmark}

\begin{document}

 \maketitle[titletotopverticalspace=0.6cm,titletoblockverticalspace=1cm, 
    innersep = 0.5cm,width = 23in]

 % --- RÆKKE 1: Intro | What? | Why? (3 kolonner) ---
 \begin{columns}
    \column{0.33}
    \block{Introduction}{Nature and visiting the great outdoors have always been an integral part of Danish life. A report from Copenhagen University examined the relationship between nature and the Danish population across the past 30 years. They found that the majority of Danes visit nature at least once a year (97.2 \% in 2024), and that Danes visit nature more frequently than previously with 62.0 \% visiting nature within the past week in 1994 compared to 71.4 \% in 2024. Participants were also asked what activities they spent time on during their latest visit. In 1994, 8.2 \% stated that they had collected berries, mushrooms, etc. This number decreased to just 4.3 \% in 2024~\cite{Nature30Years}.

Meanwhile, foraging has a number of documented health benefits, whether it is a primary lifestyle or a recreational activity. Foraging can help reduce food costs, and it has been linked to nutritional and fitness benefits~\cite{ForagingReview}. Foraging is also considered a powerful tool for connecting people to nature, and it can also help educate people about nature and local flora~\cite{UrbanForaging}. A systematic review from 2023 of foraging practices in Europe found that younger generations often lack foraging knowledge, while older generations who possess this knowledge may no longer have the energy to pass it on. This means that there may be opportunities to forge stronger connections across generations, while also reaping the benefits of being physically active and spending time in nature~\cite{ForagingReviewEurope}.

Unfortunately, barriers exist that make foraging seem inaccessible for people. One such issue is a lack of knowledge and not knowing where to look~\cite{ForagingReview}. Therefore, we set out to build a map of Denmark showing locations where different edible plant species have been previously observed.}

    \column{0.34}
    {
        \colorlet{blocktitlebgcolor}{accentone}
        \block{What?}{Observations for a curated list of edible species were sourced from the Global Biodiversity Information Facility (GBIF), covering the period from January 2023 to November 2025.
Species were manually classified into eight categories (see filters in Figure \ref{fig:filter}). 
The data cleaning process retained only records with open licenses and valid geographic coordinates for use in the final visualization.}
    }

    \column{0.33}
    {
        \colorlet{blocktitlebgcolor}{accenttwo}
        \block{Why?}{The purpose of the dashboard is to support exploratory discovery of edible plant observations in Denmark.
The visualization is needed to give insights into trends and distributions that cannot be identified using a table alone.
The dashboard supports several actions and targets:

\begin{itemize}
    \item Users \textbf{discover} spatial \textbf{features} to gain an overview of where observations are concentrated.
    \item Users \textbf{compare the distribution} of plant categories across time period.
    \item Users \textbf{summarize the distribution} of plant observations across Denmark.
    \item Users \textbf{locate attributes} when plant species are known but their location is unknown.
    \item Users \textbf{browse the distribution of plant observations} within a selected region when species are unknown.
\end{itemize}}
    }
 \end{columns}

 % --- RÆKKE 2: Full-width Figur (Uden for columns-miljøet) ---
\block{Visualization of the Foraging Tool}{
    \centering
    \setcounter{subfigcount}{1} % Starter ved (a)
    \refstepcounter{figure}     % Øger figur-nummeret (f.eks. Figur 1)

    \begin{minipage}{\linewidth}
        % --- VENSTRE KOLONNE ---
        \begin{minipage}[c]{0.72\linewidth}
            \centering
            % (a) Filter
            \begin{minipage}{\textwidth}
                \centering
                \includegraphics[width=0.95\textwidth]{figures/Filter.png}
                \mysubcaption{Filter and time selection.}
            \end{minipage}

            \vspace{1cm}

            % (b) Main Map & (c) Mini Map
            \begin{minipage}{0.60\textwidth}
                \centering
                \includegraphics[width=\textwidth]{figures/ForagingMap.png}
                \mysubcaption{Main foraging map.}
            \end{minipage}
            \hfill
            \begin{minipage}{0.37\textwidth}
                \centering
                \includegraphics[width=\textwidth]{figures/MiniMap.png}
                \mysubcaption{Detail map.}
            \end{minipage}
        \end{minipage}
        \hfill
        % --- HØJRE KOLONNE (Observations) ---
        \begin{minipage}[c]{0.24\linewidth}
            \centering
            \begin{minipage}{\textwidth}
                \centering
                \includegraphics[width=\textwidth]{figures/observations1.png}
                \mysubcaption{Obs A.}
            \end{minipage}
            
            \vspace{1.5cm}
            
            \begin{minipage}{\textwidth}
                \centering
                \includegraphics[width=\textwidth]{figures/observations2.png}
                \mysubcaption{Obs B.}
            \end{minipage}
        \end{minipage}
    \end{minipage}

    \vspace{1cm}
    \centerline{\textbf{Figure \thefigure:} Full dashboard and detailed observation views.}
}
 % --- RÆKKE 3: How? | Conclusion? (2 kolonner) ---
 \begin{columns}
    \column{0.5}
    {
        \colorlet{blocktitlebgcolor}{accentthree}
        \block{How?}{The dashboard uses multiple visualization encodings to represent the data at different levels.

\textbf{Marks}
\begin{itemize}
    \item Area marks (hexagons) to represent aggregated spatial regions
    \item Point marks to represent individual observations in the Minimap
    \item Line marks to represent aggregated temporal data in the Observation Period
\end{itemize}

\textbf{Channels}
\begin{itemize}
    \item Position encodes geographic location (map) 
    \item Color Luminance encodes observation density using sequential color scale
    \item Color Hue encodes plant categories (categorical)
\end{itemize}

Furthermore, the dashboard utilizes \textbf{Manipulate}: select and navigate, \textbf{Facet}: juxtaposition, and \textbf{Reduce}: aggregation and filter.

}
    }

    \column{0.5}
    \block{Conclusion}{Our dashboard uses principles of scientific visualization to create an interactive map that is simple for people to use. 
It utilizes different features to provide a simple and clear overview of edible plant observations around Denmark.}
 \end{columns}

 \end{document}