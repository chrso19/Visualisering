\documentclass{egpubl}
\usepackage{eurovis2023}
\EuroVisShort  % for EuroVis additional material
\usepackage[T1]{fontenc}
\usepackage{dfadobe}  
\usepackage{cite}  % comment out for biblatex with backend=biber
\BibtexOrBiblatex
\electronicVersion
\PrintedOrElectronic
\ifpdf \usepackage[pdftex]{graphicx} \pdfcompresslevel=9
\else \usepackage[dvips]{graphicx} \fi
\usepackage{egweblnk}
\usepackage{subcaption}
\usepackage{float}


\title[Foraging Map Tool]%
      {Visual Exploration of Edible Wild Flora: A Foraging-Oriented Map Tool}

\author[Kessor Chao, Chrstine Sandager Søgaard \& Anna Ellen Vedel Braaby]
{\parbox{\textwidth}{\centering Kessor Chao$^{1}$, Christine Sandager Søgaard$^{1}$ and Anna Ellen Vedel Braaby$^{1}$
        }
        \\
{\parbox{\textwidth}{\centering $^1$University of Southern Denmark, Odense, Denmark
       }
}
}


%-------------------------------------------------------------------------
\begin{document}




\teaser{
    \centering
    
    % Top wide subfigure
    \begin{subfigure}{\textwidth}
        \centering
        \includegraphics[width=0.97\textwidth]{images/Filter.png}
        \caption{Time period selector and plant category filters.}
        \label{fig:Filter}
    \end{subfigure}
    
    \vspace{0.5cm}
    
    % Bottom two subfigures side by side
    \begin{subfigure}{0.6\textwidth}
        \centering
        \includegraphics[width=\textwidth]{images/ForagingMap.png}
        \caption{Foraging map of Denmark with number of observations within the time period.}
        \label{fig:ForagingMap}
    \end{subfigure}%
    \hfill
    \begin{subfigure}{0.37\textwidth}
        \centering
        \includegraphics[width=\textwidth]{images/MiniMap.png}
        \caption{Mini map of the selected hexagon showing details of a selected observation.}
        \label{fig:MiniMap}
    \end{subfigure}
    
    \caption{Dashboard of the Foraging Map of Denmark.}
    \label{fig:Dashboard}
}


\maketitle

%-------------------------------------------------------------------------
\begin{abstract}

Foraging for edible plants offers both health and educational benefits, yet access is often limited by lack of knowledge and information on where species can be found.
We present a dashboard for exploring observations of edible plants in Denmark, using data from the Global Biodiversity Information Facility (GBIF) filtered to a curated list of edible species.
Unlike existing platforms, our tool focuses on planning foraging activities, providing an easy-to-use interface for both novices and experienced users.
The dashboard integrates a geospatial overview of Denmark, detailed regional mini maps, observation period histograms, and category filters, supporting exploration of temporal, spatial, and species-level patterns.
Our dashboard facilitates informed planning of foraging activities while maintaining potential for expansion to include additional species, hazards, or other types of natural resource collection.

\end{abstract}  
%-------------------------------------------------------------------------
\section{Introduction}
Nature and visiting the great outdoors have always been an integral part of Danish life. A report from Copenhagen University examined the relationship between nature and the Danish population across the past 30 years. They found that the majority of Danes visit nature at least once a year (97.2 \% in 2024), and that Danes visit nature more frequently than previously with 62.0 \% visiting nature within the past week in 1994 compared to 71.4 \% in 2024. Participants were also asked what activities they spent time on during their latest visit. In 1994, 8.2 \% stated that they had collected berries, mushrooms, etc. This number decreased to just 4.3 \% in 2024~\cite{Nature30Years}.

Meanwhile, foraging has a number of documented health benefits, whether it is a primary lifestyle or a recreational activity. Foraging can help reduce food costs, and it has been linked to nutritional and fitness benefits~\cite{ForagingReview}. Foraging is also considered a powerful tool for connecting people to nature, and it can also help educate people about nature and local flora~\cite{UrbanForaging}. A systematic review from 2023 of foraging practices in Europe found that younger generations often lack foraging knowledge, while older generations who possess this knowledge may no longer have the energy to pass it on. This means that there may be opportunities to forge stronger connections across generations, while also reaping the benefits of being physically active and spending time in nature~\cite{ForagingReviewEurope}.

Unfortunately, barriers exist that make foraging seem inaccessible for people. One such issue is a lack of knowledge and not knowing where to look~\cite{ForagingReview}. Therefore, we set out to build a map of Denmark showing locations where different edible plant species have been previously observed.


%-------------------------------------------------------------------------
\section{Related Work}
Describe work that relates to your project, and why your work is different from what exists. In reference to the temperature change map (Figure~\ref{fig:tchange}) there may be other projects that do visualize similar stuff.

% -----------------------

\section{Data}
The dataset used in this work was obtained from the Global Biodiversity Information Facility (GBIF) \cite{gbif2026}.
We extracted observations for a list of edible species, using geographic coordinates and dates to map occurrences.
Each record contains additional metadata, such as observer, institution, and occurrence details. 
These metadata could be incorporated in future iterations to enhance the dashboard.
This structured dataset provides a reliable foundation for analysis and visualization while allowing for further expansion and refinement over time.
%-------------------------------------------------------------------------
\section{Design}
Describe the design of your system, maybe you can organize the section based on the What, Why and How questions... You could, e.g., also refer to the Information Seeking Mantra~\cite{InformationSeeking}, if your interface operates accordingly.


The visualization tool supports exploratory discovery of edible plant species in Denmark by using spatial, temporal, and categorical features in the dataset.
Our tool follows the Visual Information Mantra \cite{InformationSeeking}, starting with a geospatial overview of Denmark, followed by zooming in on hexagonal regions and filtering by categories and date range, and finally providing details-on-demand through the minimap and observation periods views.

The tool consists of the following features/views:

\paragraph{Map of Denmark}
The choropleth map allows for initial exploration of edible plant observations. 
Since individual observations are too numerous to be meaningfully interpreted, we decided to aggregate the observations into spatial regions, using hexagons. Each hexagon represents observation density using luminance color scale.
Using luminance color scale on the map, gives the user a quick interpretation of where observations are concentrated and where data is sparse across regions. In addition, it helps the user to decide which areas to explore further.
Selecting a hexagon triggers the display of a mini map with detailed information within the selected region, enabling a smooth transition from overview to detailed inspection.


\paragraph{Mini map}
While the choropleth map supports high-level spatial exploration, the mini map provides a detailed local exploration of a selected region. 
The purpose of the mini map is to allow users to inspect individual observation within a selected hexagon. By highlighting the selected hexagonal region in the mini map, we can distinguish observations within the hexagon from the surrounding observations.
The mini map uses point markers to represent individual observations. The categories are encoded using color hue which is consistent with the category filter menu, allowing users to distinguish categories at local scale.
The mini map supports details-on-demand by allowing users to hover over or select individual observations to see species information and observation dates.
Finally, placing the mini map next to the map of Denmark, we allow users to see the individual observations while maintaining awareness of the selected region in the broader spatial context.


\paragraph{Observation per period}
The observation period visualization is designed to help users understand when observations occur. While the geospatial view shows where observations are concentrated, they do not convey temporal patterns such as seasonality.
The "observation per period" view allows users to identify periods with higher or lower observation activity across years, months, and days.
Observations are aggregated by their categories and into time bins using a stacked histogram. Time is encoded along the x-axis, while the number of observation is encoded along the y-axis. 
Plant categories are distinguished using color hue, consistent with the rest of the visualization.
Users can select a specific time period \(stacked bar\) to access more detailed views, enabling a drill-down from monthly to daily observations.


\paragraph{Filter by date range}
Date picker defines the temporal scope of the tool. Observation data spans from 2023-01-01 to 2025-11-16. By filtering the time period, the users can interpret the plant observations within a temporal context. 
The date range filter is applied globally across the tool, ensuring that all visualizations update simultaneously and remain temporally consistent.

\paragraph{Filter menu for category}
The category filter menu supports focused exploration of the dataset by allowing users to select or deselect entire plant categories or individual species. 
This enables users to explore subsets of the data relevant to their interest. Filter selections are applied consistently across all views and can be adjusted continuously, supporting an exploratory workflow.




%-------------------------------------------------------------------------
\section{Results}
What are the results of your project, e.g., what are the findings in Figure~\ref{fig:tchange}?

\begin{figure}[tbp]
  \includegraphics[width=\linewidth]{images/tchange.png}
  \caption{\label{fig:tchange}%
We can see the change of temperature by Celsius degrees for 500 weather stations across the world in the past 100 years. 
  }
\end{figure}

%-------------------------------------------------------------------------
\section{Future Work}
While the dashboard works as intended and contains the elements, we wanted, there is still room for improvement.
One potential improvement would be to include a feature to select one's favorite species, so it is not necessary to make the same selection each time.
It would also be useful for the users to share the mini map, enabling access to the information, while being out foraging.
Since the dashboard is targeted at both novices and experienced foragers, it would also be convenient to include information on any poisonous species that look like the species being foraged for,
as well as guidance on pests such as ticks or mosquitos. 
The foraging map could also be expanded further to include fishing or hunting grounds, which would enable users to become even more self-sufficient.
Finally, conducting user testing and collecting feedback from users could help validate design choices and identify improvements to the interface.

%-------------------------------------------------------------------------
\section{Conclusion}
Our dashboard uses principles of scientific visualization to create an interactive map that is simple for people to use. 
It utilizes different features to provide a simple and clear overview of edible plant observations around Denmark.

%-------------------------------------------------------------------------
% References
\bibliographystyle{eg-alpha-doi}
\bibliography{references}

\end{document}