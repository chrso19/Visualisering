\begin{figure}[ht]
\centering

\begin{subfigure}[b]{0.42\textwidth}
  \includegraphics[width=1\linewidth]{images/Observations1.png}
  \caption{Histogram of observations made within the hexagon selected for the chosen time period.}
  \label{fig:Observations1} 
\end{subfigure}

\medskip % insert a bit of vertical whitespace
\begin{subfigure}[b]{0.42\textwidth}
  \includegraphics[width=1\linewidth]{images/Observations2.png}
  \caption{Histogram containing greater observation details for a shorter time period.}
  \label{fig:Observations2}
\end{subfigure}

\caption{Observation period tab for the selected hexagon of the foraging map.}

\end{figure}

The resulting dashboard consists of a number of elements. 
At the top of the dashboard, there is the time period selector and filter for plant categories, which can be seen in Figure~\ref{fig:Filter}. 
There is also a dropdown menu for each plant category, where it is possible to select specific plant species. 
Based on the selected time period and plant species, a map of Denmark is generated with hexagons to group observations, which can be seen in Figure~\ref{fig:ForagingMap}. 
As one's mouse moves across the map, a hover box appears containing the observations by category in the selected hexagon and the total number of observations.
When a hexagon is selected, a red circle appears to visualize the area selected.  

Once a hexagon has been selected, a mini map is generated to show the area in greater detail, as can be seen in Figure~\ref{fig:MiniMap}. 
It is possible to hover over the different observations and see the English name of the plant observed along with its category and the date of the observation. 
When an observation on the mini map is selected, a number of details show up, which include the English, Latin, and Danish names, the date of the observation, and the specific coordinates. 

It is also possible to switch to the "Observation Periods" tab, where a histogram is generated based on the selected time period, plant categories, and hexagon from the foraging map. 
Figure~\ref{fig:Observations1} shows the spread of observations in the hexagon. 
There is also a dropdown menu, which shows the specific plant species that were observed. 
If more details are needed, it is also possible to select a bin on the histogram. 
If one is selected, a new histogram is generated, which shows the observations of the selected bin in greater detail. 
This can be seen in Figure~\ref{fig:Observations2}. 
It is also possible to go back to the previous histogram.

The purpose of the dashboard is that is simple and easy to use, regardless of one's proficiency with technology. 
It is easy to get a quick overview of where more plants are observed in Denmark, and which categories these observations belong to. 
As one becomes more acquainted with the dashboard, it is possible to become more specific and get more details of the observations. 
The colors of the plant categories are kept consistent throughout all elements of the dashboard. 
The colors were tested to ensure that they would also be differentiable by people with colorblindness. 

