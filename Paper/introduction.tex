Nature and visiting the great outdoors have always been an integral part of Danish life. A report from Copenhagen University examined the relationship between nature and the Danish population across the past 30 years. They found that the majority of Danes visit nature at least once a year (97.2 \% in 2024), and that Danes visit nature more frequently than previously with 62.0 \% visiting nature within the past week in 1994 compared to 71.4 \% in 2024. Participants were also asked what activities they spent time on during their latest visit. In 1994, 8.2 \% stated that they had collected berries, mushrooms, etc. This number decreased to just 4.3 \% in 2024~\cite{Nature30Years}.

Meanwhile, foraging has a number of documented health benefits, whether it is a primary lifestyle or a recreational activity. Foraging can help reduce food costs, and it has been linked to nutritional and fitness benefits~\cite{ForagingReview}. Foraging is also considered a powerful tool for connecting people to nature, and it can also help educate people about nature and local flora~\cite{UrbanForaging}. A systematic review from 2023 of foraging practices in Europe found that younger generations often lack foraging knowledge, while older generations who possess this knowledge may no longer have the energy to pass it on. This means that there may be opportunities to forge stronger connections across generations, while also reaping the benefits of being physically active and spending time in nature~\cite{ForagingReviewEurope}.

Unfortunately, barriers exist that make foraging seem inaccessible for people. One such issue is a lack of knowledge and not knowing where to look~\cite{ForagingReview}. Therefore, we set out to build a map of Denmark showing locations where different edible plant species have been previously observed.